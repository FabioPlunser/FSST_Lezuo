\newpage
\section{Aufgabenstellung}
    \begin{lstlisting}[language=Python, style=Stylepython, caption=Angabge, captionpos=b, label=Angabge]
# deciphers to "Schoene Crypto Welt" with IV=BBBBBBBBBBBBBBBB and key=BBBBBBBBBBBBBBBB aes128-cbc
cyphertext = "AAE365272C81078AB6116B361831D0F6A5D3C8587E946B530B7957543107F15E"
bc = binascii.unhexlify(cyphertext)
data = b'D' + bytes([len(bc)]) + binascii.unhexlify(cyphertext) + b'X'  
\end{lstlisting}
    
\noindent  Der cyphertext soll entschlüsselt \anfuehrung{Schöne Crypto Welt} bedeuten. 
Um dies zu überprüfen kann \href{OpenSSL}{https://www.openssl.org/} verwendet werden.\\

\noindent Schreiben Sie ein Programm das unter Verwendnung von openssl obige Aussage überprüft, 
verbessen Sie ihr Program in dem Sinne dass sie key/iv/plaintexte/ciphertexte als Argumente/Dateien/Usereingaben verarbeiten.

\subsection*{Hinweise}
\begin{itemize}
    \item Sie benötigen die openssl Biblotheksheader, unter Ubuntu 20.04 können Sie diese installieren via:
    \begin{lstlisting}[language=Python, style=Stylepython, caption=Angabge, captionpos=b, label=Angabge]
$ sudo apt install libssl-dev
    \end{lstlisting}
    
    \item em Linker muss mitgeteilt werden dass sie in Ihrem Programm Funktionen verwenden die in einer externen Bibliothek bereit liegen, verwenden 
    sie dazu das flag -l (klein-L) und den Namen der Bibliothek OHNE das führende lib. 
    openssl besteht aus mehreren Bibliotheken, die für AES notwendingen Funktionen befinden sich in libcrypto.
    \begin{lstlisting}[language=Python, style=Stylepython, caption=Angabge, captionpos=b, label=Angabge]
$ gcc my_code.c -lbibliothek -o my_executable
    \end{lstlisting}
    Sie können sich die gelinkten Bibliotheken dann via ldd Kommando ansehen
    \begin{lstlisting}[language=Python, style=Stylepython, caption=Angabge, captionpos=b, label=Angabge]
$ ldd my_executable
    \end{lstlisting}
\end{itemize}
